\documentclass[12pt,a4paper]{article}
\usepackage{amsfonts, amssymb, amsmath}
\usepackage{fullpage}
\usepackage{parskip} % skip a line instead of indenting
% \usepackage{graphicx} % for inserting images
\usepackage{amsthm}
\usepackage{xcolor}
\usepackage{tikz} % for plot

\newtheorem*{rem}{Remark}

\title{Measures}
\author{R4 Cheng}
\date{\today}

\newcommand{\Remark}[1]{
  \begin{rem}
    \color{cyan}
    #1
  \end{rem}
}

\begin{document}
\maketitle

\subsection*{For Continuous Data}

\begin{enumerate}
    \item Central Tendency
    \item Variability or Dispersion
    \item Skewness
    \item Kurtosis
\end{enumerate}

\subsection*{Central Tendency}

Common central tendency measures: mean, median, mode (the most frequent value)

\subsubsection*{Mean}

Sample Mean:
\[
\bar{x} = \frac{\sum x_i}{n}
\]

Population Mean:
\[
\mu = \frac{\sum X_i}{N}
\]

\subsubsection*{Median}

Sample Median: $\tilde{x}$

Population Median: $\eta$ (eta)

\subsection*{Dispersion or Variability}

4 common measures of dispersion:

\begin{enumerate}
    \item Range: $R = \text{max} - \text{min}$
    \item Variance: population $\sigma^2 = \frac{\sum (X_i - \mu)^2}{N}$, sample $s^2 = \frac{\sum (x_i - \bar{x})^2}{n-1} = \frac{\sum x_i^2 - \frac{(\sum x_i)^2}{n}}{n-1} $
    \item Standard Deviation: population $\sigma = \sqrt{\sigma^2}$, sample $s = \sqrt{s^2}$
    \item Coefficient of Variation (CV): population $CV = \frac{\sigma}{\mu} \times 100\%$, sample $CV = \frac{s}{\bar{x}} \times 100\%$ (no unit)
\end{enumerate}

\Remark{Why use $n-1$? becuase it is proved to be more accurate.}

Disadvantages of Range: sensitive to outliers

Variance represents the distance from the mean

Variance and Standard Deviation are absolute measures of dispersion (about mean),
while Coefficient of Variation is a relative measure of dispersion (about mean).

\subsection*{Skewness}

Aka. shape of the distribution

3 types of skewness:

\begin{enumerate}
    \item Symmetrical: mean = median = mode
    \item Right Skewness or Positive Skewness: mean $>>$ median
    \item Left Skewness or Negative Skewness: mean $<<$ median
\end{enumerate}

Skewness Coefficient ($g_1$): $g_1 = \frac{\frac{\sum (X_i - \bar{x})^3}{n-1}}{s^3}$

\begin{enumerate}
    \item $g_1 = 0$: symmetrical 
    \item $g_1 > 0$: right skewness
    \item $g_1 < 0$: left skewness
\end{enumerate}

\subsection*{Kurtosis}

\[ g_2 = \frac{\frac{\sum(x_i - \bar{x})^4}{n-1}}{s^4} - 3\]

\begin{enumerate}
    \item $g_2 = 0$: meso-kurtic
    \item $g_2 > 0$: lepto-kurtic (more peaked)
    \item $g_2 < 0$: platy-kurtic (less peaked)
\end{enumerate}



\end{document}