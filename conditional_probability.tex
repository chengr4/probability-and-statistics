\documentclass[12pt,a4paper]{article}
\usepackage{amsfonts, amssymb, amsmath}
\usepackage{fullpage}
\usepackage{parskip} % skip a line instead of indenting
% \usepackage{graphicx} % for inserting images
\usepackage{amsthm}
\usepackage{xcolor}
\usepackage{tikz} % for plot

\newtheorem*{rem}{Remark}

\title{Conditional Probability}
\author{R4 Cheng}
\date{\today}

\newcommand{\Remark}[1]{
  \begin{rem}
    \color{cyan}
    #1
  \end{rem}
}

\begin{document}
% \tableofcontents % generate a table of contents
\maketitle

\textbf{Def.} $P(A \mid B) = \frac{P(A \cap B)}{P(B)}$ is the probability of $X$ given $Y$.

\begin{enumerate}
    \item $X$: event of interest
    \item $Y$: event that has been observed (condition)
\end{enumerate}

\Remark{if outcome $X$ and $Y$ are mutually exclusive, then $P(X|Y) = 0$}

\[ P(A \cap B) = P(A \mid B) \cdot P(B) = P(B \mid A) \cdot P(A) \]

\subsection*{Total Probability Theorem}

If $C_1$, $C_2$, ..., $C_n$ are a set of \textbf{mutually exclusive and exhaustive events}, i.e. $C_1 \cup C_2 \cup C_3 \cup ... \cup C_n = S$, then for any event $A$:

\[ P(A) = \sum_{i=1}^{n} P(A \mid C_i) P(C_i) =  P(A \mid C) P(C) + P(A \mid \bar{C}) P(\bar{C}) \]

\textbf{Benefit}: It can separate an event into a set of mutually exclusive and exhaustive sub-events.

\subsection*{Bayes' Theorem}

\[ P(A \mid B) = \frac{P(B \mid A) P(A)}{P(B)} \]


\textbf{When to use?} Help us infer unknown conditional probabilities based on the given conditions.

$\Rightarrow$

\[P(A \mid B) = \frac{P(A \cap B)}{P(B)} \quad \text{(insert Total Probability Theorem)}\]

\[ = \frac{P(B \mid A)P(A)}{P(B \mid A)P(A) + P(B \mid \bar{A})P(\bar{A})}\]

\end{document}